%%%%%%%%%%%%%%%%%%%%%%%%%%%%%%%%%%%%%%%%%%%%%%%%%%%%%%%%%%%%%%%%%%%%%%%%%%%%%%%
%% Copyright (c) 2010, 2023 Contributors to the Eclipse Foundation
%%
%% See the NOTICE file(s) distributed with this work for additional
%% information regarding copyright ownership.
%%
%% This program and the accompanying materials are made available under the terms
%% of the MIT License which is available at https://opensource.org/licenses/MIT
%%
%% SPDX-License-Identifier: MIT
%%%%%%%%%%%%%%%%%%%%%%%%%%%%%%%%%%%%%%%%%%%%%%%%%%%%%%%%%%%%%%%%%%%%%%%%%%%%%%%

In this report, the \ecorelang{} language is defined. The \ecorelang{}
language is defined using a so-called \emph{conceptual model}, also known as
\emph{metamodel} by the Object Management Group (OMG). A metamodel represents
concepts (entities) and relationships between them. The \ecorelang{} metamodel
is described using (Ecore) class diagrams~\cite{Steinberg4:EMFBook09}, where
classes represent concepts, and associations represent relationships between
concepts. Static semantic constraints and relations that cannot be represented
using class diagrams are stated in the class documentation of the metamodel.
The metamodel and the accompanying constraints are used primarily to formalize
the syntax of the internal (implementation) representation of the language.
