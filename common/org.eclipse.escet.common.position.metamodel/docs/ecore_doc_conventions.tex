%%%%%%%%%%%%%%%%%%%%%%%%%%%%%%%%%%%%%%%%%%%%%%%%%%%%%%%%%%%%%%%%%%%%%%%%%%%%%%%
%% Copyright (c) 2010, 2021 Contributors to the Eclipse Foundation
%%
%% See the NOTICE file(s) distributed with this work for additional
%% information regarding copyright ownership.
%%
%% This program and the accompanying materials are made available under the terms
%% of the MIT License which is available at https://opensource.org/licenses/MIT
%%
%% SPDX-License-Identifier: MIT
%%%%%%%%%%%%%%%%%%%%%%%%%%%%%%%%%%%%%%%%%%%%%%%%%%%%%%%%%%%%%%%%%%%%%%%%%%%%%%%

Each (sub-)package is described in a separate section. An informal description
of the package is followed by the Uniform Resource Identifier (URI) of the
package, the namespace prefix, and a list of all the direct sub-packages. All
classifiers defined in the package are described in sub-sections. First the
data types are described, then the enumerations, and finally the classes. The
data types are ordered lexicographically, as are the enumerations and classes.

For data types, an informal description of the data type is followed by the
name of the data type, the instance class name, basetype, and the (regular
expression) pattern.

For enumerations, an informal description of the enumeration is followed by
information about the enumeration literals, which are ordered
lexicographically. For each enumeration literal, a short informal description
is included. The default value of the enumeration (the default literal), is
indicated as well.

For classes, an informal description of the class is followed by the
inheritance hierarchy. Note that all classes that do not have an explicit
supertype in the Ecore, implicitly inherit from
\emph{EObject}\footnote{Actually, in the implementation,
\emph{org.eclipse.emf.ecore.EObject} and all classes from metamodels are
interfaces. Implementation classes implement the interfaces and have names
ending in \emph{Impl}. E.g. \emph{org.eclipse.emf.ecore.impl.EObjectImpl}
implements \emph{org.eclipse.emf.ecore.EObject}.}. Therefore, all the
inheritance hierarchies start in \emph{EObject}. The inheritance hierarchies
are followed by a listing of all the directly derived classes of the class.
Finally, all the structural features of the class are listed, including the
inherited ones. The structural features of the supermost type are listed
first, and the ones of the actual class are listed last. Secondary ordering
is lexicographical.

For each structural feature, the type is indicated (`attr' for attributes,
`ref' for references, and `cont' for containment references). This is followed
by the name of the structural feature, the multiplicity, a colon, and the
type. If the structural feature is inherited from a supertype, that is
indicated as well. Finally, an informal description of the structural feature
is provided.
