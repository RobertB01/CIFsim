%%%%%%%%%%%%%%%%%%%%%%%%%%%%%%%%%%%%%%%%%%%%%%%%%%%%%%%%%%%%%%%%%%%%%%%%%%%%%%%
%% Copyright (c) 2010, 2024 Contributors to the Eclipse Foundation
%%
%% See the NOTICE file(s) distributed with this work for additional
%% information regarding copyright ownership.
%%
%% This program and the accompanying materials are made available under the terms
%% of the MIT License which is available at https://opensource.org/licenses/MIT
%%
%% SPDX-License-Identifier: MIT
%%%%%%%%%%%%%%%%%%%%%%%%%%%%%%%%%%%%%%%%%%%%%%%%%%%%%%%%%%%%%%%%%%%%%%%%%%%%%%%

Metamodels are represented using Ecore class diagrams, which are very similar
to UML class diagrams. In Ecore class diagrams,
\emph{classifiers} represent concepts, and \emph{associations} represent
relationships between concepts. There are two kinds of classifiers, namely
\emph{data types} and \emph{classes}.

Data types are used for simple types, whose details are not modeled as classes.
Data types are identified by a name. Examples of data types include booleans,
numbers, strings (optionally restricted using regular expressions), and
enumerations.

A class is also identified by its name, and can have a number of structural
features, namely attributes and references. Classes allow \emph{inheritance},
giving them access to the structural features of their supertypes/basetypes.

\emph{Attributes} are identified by name, and they have a data type.
Associations between classes are modeled by \emph{references}. Like attributes,
references are identified by name and have a type. However, the type is the
class at the other end of the association. A reference specifies lower and
upper bounds on its multiplicity. The multiplicity indicators that can be used
are shown in Table~\ref{tbl:multiplicity}.
\begin{table}
\caption{Multiplicity indicators}\label{tbl:multiplicity}
\begin{center}
\begin{tabular}[htb]{@{}|r|@{\quad} l |}
    \hline
    \textbf{Indicator} & \textbf{Meaning} \\
    $n$    & Exactly $n$ (where $n\ge 1$), default notation \\
    $n..n$ & Exactly $n$ (where $n\ge 1$), alternative notation \\
    $n..m$ & $n$ up to and including $m$
             (where $n\ge 0$, $m\ge 1$, and $m>n$) \\
    $n..*$ & $n$ or more (where $n\ge 0$) \\
    \hline
\end{tabular}
\end{center}
\end{table}
Finally, a reference specifies whether it is being used to represent a
stronger type of association, called \emph{containment}.

Graphically, data types are depicted as rectangles. The rectangles have a
yellow background. The data type name is shown at the top inside the
rectangle. The Java class name is shown below it. Enumerations differ
slightly. They have a green background. Instead of the Java class name,
the enumeration literals are listed below the name of the enumeration.

Classes are depicted as rounded rectangles with a yellow background. The
class name is shown at the top inside the rectangle. Abstract classes have
a grey background, and the class name is shown in italic font. The names,
types and multiplicity of the attributes are shown inside the rectangle.
References for which the target class is not part of the diagram, are
listed as well. Features from base classes are listed using a grey font.

Tables~\ref{tbl:ecore-classifier-icons} and \ref{tbl:ecore-feature-icons}
shows the various icons used in Ecore class diagrams for classifiers and
features.

\begin{table}
\caption{Ecore diagram classifier icons}\label{tbl:ecore-classifier-icons}
\begin{center}
\begin{tabular}[htb]{@{}|c|@{\quad} l |}
    \hline
    \textbf{Icon} & \textbf{Meaning} \\
    \includegraphics{figures/ecore_icon_datatype.png} & Data type \\
    \includegraphics{figures/ecore_icon_enum.png} & Enumeration \\
    \includegraphics{figures/ecore_icon_class.png} & Class \\
    \includegraphics{figures/ecore_icon_class_abstr.png} & Abstract class \\
    \hline
\end{tabular}
\end{center}
\end{table}

\begin{table}
\caption{Ecore diagram feature icons}\label{tbl:ecore-feature-icons}
\begin{center}
\begin{tabular}[htb]{@{}|c|@{\quad} l |}
    \hline
    \textbf{Icon} & \textbf{Meaning} \\
    \includegraphics{figures/ecore_icon_attr_01.png} &
      Attribute with multiplicity $[0..1]$ \\
    \includegraphics{figures/ecore_icon_attr_11.png} &
      Attribute with multiplicity $[1..1]$ \\
    \includegraphics{figures/ecore_icon_ref_01.png} &
      Reference with multiplicity $[0..1]$ \\
    \includegraphics{figures/ecore_icon_ref_11.png} &
      Reference with multiplicity $[1..1]$ \\
    \includegraphics{figures/ecore_icon_ref_0x.png} &
      Reference with multiplicity $[0..*]$ \\
    \includegraphics{figures/ecore_icon_ref_1x.png} &
      Reference with multiplicity $[1..*]$ \\
    \hline
\end{tabular}
\end{center}
\end{table}

Inheritance relations are depicted as arrows between two classes with a
(non-solid) triangle on the side of the superclass. A reference is depicted as
an arrow between two classes, labeled with its name and its multiplicity. A
containment reference is depicted with a solid diamond at the side of the
containing class.
